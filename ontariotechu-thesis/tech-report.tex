% note that the "draftcls" or "draftclsnofoot", not "draft", option
% should be used if it is desired that the figures are to be displayed in
% draft mode.

\documentclass[12pt,letterpaper]{article}

% Some useful packages
\usepackage{lipsum} 
\usepackage{changepage}
\usepackage{setspace}
\usepackage{multicol}
\usepackage{graphicx}
\usepackage[export]{adjustbox}
\usepackage{subcaption}
\usepackage{rotating}
\usepackage{footnote}
\usepackage{caption}
\usepackage{amsmath,amssymb}
\usepackage{algorithm}
\usepackage{listings}
\usepackage[noend]{algpseudocode}
\usepackage[shortcuts]{extdash}
\usepackage{ctable}
\usepackage{dcolumn}
\usepackage{multirow}
\usepackage{color} 
\usepackage{cite}
\usepackage{url}
\usepackage[utf8]{inputenc} 
\usepackage{booktabs}
\usepackage[toc,page]{appendix}
\usepackage{setspace}
% Page layout
% Final submission
\usepackage[top=1in, bottom=1in, left=.75in, right=.75in]{geometry}
% Draft mode that allows for larger space on the right margin
% for notes etc.
% \usepackage[top=1in, bottom=1.5in, left=.75in, right=3.25in]{geometry}

\begin{document}

\title{Tech Report}

\author{Thing 1 and Thing 2\\
Faculty of Science\\
Ontario Tech University\\
}
\date{\today}

\maketitle

% Often do not need the following.  Leave them commented.
%% \tableofcontents \newpage
%% \listoffigures \newpage
%% \listoftables \newpage

% Uncomment the following if you want more space between the 
% lines.  This is useful in the draft mode.
% \setstretch{2.5}


\begin{abstract}
  \lipsum[1]
\end{abstract}

\section{Introduction}
\lipsum[2-4]

\section{Another Section}
\lipsum[2-4]

\newpage
\begin{appendices}

\section{A Section in the Appendix}
\lipsum[2-5]

\end{appendices}
\newpage

\addcontentsline{toc}{chapter}{Bibliography}

\bibliographystyle{plain}

\bibliography{thesis} % A dummy thesis.bib file.

\end{document}
